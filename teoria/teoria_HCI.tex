%----------------------------------------------------------------------------------------
%	PAQUETES Y CONFIGURACIÓN DEL DOCUMENTO
%----------------------------------------------------------------------------------------

%%% Configuración del papel.
\documentclass[12pt]{article} 
\usepackage[utf8]{inputenc}
\usepackage[catalan]{babel}
\usepackage{blindtext}
\usepackage{graphicx} 
\usepackage[margin=3cm]{geometry}
\usepackage{subcaption}
\usepackage{listings}
\usepackage{caption}
\usepackage{hyperref}
\usepackage{amsmath}

\newcommand{\horrule}[1]{\rule{\linewidth}{#1}} % Create horizontal rule command with 1 argument of height

\title{
  \normalfont \normalsize 
  \textsc{FIB: Interacció i Disseny d'Interfícies} \\ [25pt] 
  \horrule{0.5pt} \\[0.4cm] % Thin top horizontal rule
  \huge Resum per l'examen final \\ % The assignment title
  \horrule{2pt} \\[0.5cm] % Thick bottom horizontal rule
}

\author{Ismael El Basli}
\usepackage{xcolor}

\definecolor{codegreen}{rgb}{0,0.6,0}
\definecolor{codegray}{rgb}{0.5,0.5,0.5}
\definecolor{codepurple}{rgb}{0.58,0,0.82}
\definecolor{backcolour}{rgb}{0.95,0.95,0.92}
\lstdefinestyle{mystyle}{
    backgroundcolor=\color{backcolour},   
    commentstyle=\color{codegreen},
    keywordstyle=\color{magenta},
    numberstyle=\tiny\color{codegray},
    stringstyle=\color{codepurple},
    basicstyle=\ttfamily\footnotesize,
    breakatwhitespace=false,         
    breaklines=true,                 
    captionpos=b,                    
    keepspaces=true,                 
    numbers=left,                    
    numbersep=5pt,                  
    showspaces=false,                
    showstringspaces=false,
    showtabs=false,                  
    tabsize=2
}
\lstset{style=mystyle}

%----------------------------------------------------------------------------------------
%	DOCUMENT
%----------------------------------------------------------------------------------------

\begin{document}
    \maketitle
    \newpage
    \tableofcontents
    \newpage
    \section{Introducció}
    La part d'usabilitat i principis de disseny, a part de les lleis de percepció, ja van entrar al primer parcial, motiu pel qual no apareixen en
    aquest resum. L'objectiu d'aquest document és resumir de forma breu els conceptes d'HCI (\textit{Human Computer Interaction}) més importants
    per l'examen final.

    \section{Disseny d'Interacció i Avaluació}
    \subsection{Fonaments de la interacció bàsica a l'interfície d'usuari}
    \subsubsection{Teoría de la informació}
    Cal tenir clar els següents conceptes:
   
    Si \textit{d} és un dispositiu que produeix els símbols A, B, C i D amb la mateixa probabilitat:
    \begin{itemize}
        \item El total de símbols és M = 4
        \item Cada cop que un símbol es produeix no tenim certesa de quin símbol es generarà
        \item La probabilitat de que un símbol aparegui és de $P = \frac{1}{M} = \frac{1}{4}$
    \end{itemize}

    Aquesta \textbf{incertesa} que tenim la mesurem amb $\log_{2}(M)$. En el nostre cas, el resultat són 2 bits.
    Aquesta mesura resulta ser molt bona, ja que per exemple amb M = 1 (només un símbol), no hi ha incertesa ja que
    $\log_2(1) = 0$.
    
    \hfill \break
    També ens podria sortir a l'examen un exercici amb dos dispositius d i e, un amb sortides A, B i C i l'altre amb
    1, 2, respectivament. En aquest cas, el total de símbols s'obté fent totes les \textbf{combinacions} possibles (M = 6).
    La incertesa és per tant la suma d'incerteses: $log_2(6) = log_2(2) + log_2(3)$.


    \hfill \break
    Si reescribim la fòrmula de l'incertesa (incertesa = $log_2(M)$, on M = total de símbols), tenint en compte que 
    $P = \frac{1}{M}$ obtenim:
    \begin{itemize}
        \item $log_{2}(M) = log_{2}((\frac{1}{M})^{-1}) = log_{2}(P^{-1}) = -log_{2}(P)$
    \end {itemize}
    Aquest $-log_{2}(P)$ rep el nom de \textbf{sorpresa}, i es correspon amb la sorpresa de trobar un símbol determinat.
    Si tenim M símbols amb probabilitat $p_{i}$, s'ha de complir $\sum_{i=1}^{M}p_{i} = 1$.

    \newpage
    Tenint en compte tots els conceptes mencionats abans, s'obté l'equació de l'\textbf{entropia de Shanon}, que mesura la quantitat
    d'informació que s'espera rebre, on H és l'entropia, N és el nombre d'alternatives i $p_{i}$ la probabilitat de
    l'alternativa iéssima: \hfill \break

    $H = \sum_{i=1}^{N}p_{i}log_{2}(\frac{1}{p_{i}}) = -\sum_{i=1}^{N}p_{i}log_{2}(p_{i})$

    \hfill \break

    Si aquesta entropia ens dóna un resultat d'1, això significa que la font necessita una mitja d'1 bit per símbol.
    En aquest punt, cal destacar que és possible tenir menys d'1 bit per símbol, jugant amb les codificacions dels símbols
    i fent que aquells que apareixen més tinguin una codificació amb menys bits. (Com a detall, un bon algorisme per aquest objectiu 
    és la codificació Huffman, que construeix la solució utilitzant un arbre binari). Cal tenir en compte també que no s'han considerat
    les interferències, i que les equacions es compliquen una mica més considerant-ho. ($R = H(x) - H_{y}(x)$, on $H_y(x)$ 
    correspon a l'equivocació, l'informació requerida per quantificar l'error.)
    

    \subsubsection{Llei de Hick-Hyman}
    Descriu el temps de decisió humà com a funció del contingut d'informació transmés per un estímul visual.
    \hfill \break \\
    El temps per prendre una decisió (\textit{Reaction Time}) és: $RT = a + bH_T$, on a i b són constants i $H_T$ l'informació transmesa.
    \\
    Com que $H_T = log_2(n+1)$, on n són les alternatives equiprobables, llavors: $RT = a + blog_2(n+1)$.

    \hfill \break
    Hick va comprobar la correctesa d'aquesta fòrmula, i Hyman va trobar que també servia per alternatives no equiprobables.
    
    
    \begin{itemize}
        \item + Per les localitzacions més usades i freqüents, el temps de decisió s'ajusten a la llei de Hick-Hyman
        \item - Per les localitzacións no apreses, s'ajusten a un cerca lineal (NO logarítmica)
        \item - A més, els usuaris nous també busquen amb temps de cerca lineals. 
    \end{itemize}

    En conclusió, la llei de Hick-Hyman no serveix per tots els casos, els temps de decisió humans no sempre s'ajusten a una corva 
    logarítmica que és funció dels estímuls visuals.
    \newpage

    \subsubsection{Llei de Fitts}
    Estableix una relació lineal entre el temps de moviment (MT) i la dificultat de la tasca.
    \\
    $MT = a + b ID$, on ID és l'índex de dificultat, a correspon als temps d'inici/parada en segons i la constant b 
    és inherent a la velocitat del dispositiu. Pel que fa a ID, $ID = log_2{\frac{2A}{W}}$, on A és l'amplitud del moviment i 
    W l'amplada de l'objectiu. És a dir, com més amplitud de moviment més dificultat i com més amplada de l'objectiu menys dificultat.
    \\
    Els resultats experimentals mostren que sí que hi ha una relació lineal entre el MT i l'ID. La situació de màxima dificultat seria
    una W molt petita i una A molt gran. A més, cal mencionar que han aparegut diverses fòrmules que deriven d'aquesta cambiant certs paràmetres,
    per adaptar-la a altres situacions possibles. (en totes elles, l'ID és funció de la distància cap a l'objectiu i la seva mida). D'aquestes, la que ens podria
    sortir en un exercici és la variant de MacKenzie's (una de les més acceptades): $MT = a + blog_2(\frac{D}{W} + 1)$, on D és la distància del moviment, i els demés paràmetres el mateix que abans.

    \begin{itemize}
        \item + S'ha demostrat la validesa de la llei de Fitts en múltiples configuracions i dispositius.
        \item + La llei de Fitts és un bon model predictiu del moviment humà.
        \item - La llei de Fitts no modela bé objectius petits i pantalles tàctils (hi ha noves fòrmules per això, com FFitts).
    \end{itemize}

    \subsubsection{Lleis de creuament i direcció (\textit{Crossing and Steering: Continuous Gestures})}
    Pel que fa a la llei de Crossing, només cal recordar que creuar un objecte és més fàcil que donar-li doble click, ja que els llapissos
    o els dits naturalment condueixen a gestos de creuament. Per tant, creuar objectes pot ser una bona alterativa per usuaris amb dificultats
    pels clicks/doble clicks (a més que amb un mateix moviment és poden creuar múltiples objectes). En base a experiments, el creuament en comparació amb
    apuntar i clickar, aconsegueix resultats similars (o + ràpids) i l'error és menor, però es torna més difícil quan la distància entre objectius és petita.
    
    \hfill \break
    Per altra banda, la llei de Steering ens diu que navegar per un camí de mida restringida és útil a les interfícies d'usuari (menus imbricats, etc). Els resultats
    mostren que la llei de Steering és aplicable a diferents configuracions.

    \newpage
    \subsection{Llei de Fitts al disseny d'interfícies d'usuari}
    Com ja s'ha mencionat abans, la llei de Fitts prediu amb precisió el moviment d'apuntar, mostrant que a major distància més difícil és la selecció.
    Per tant, podem tenir en compte la llei per modificar la nostra UI, cambiant l'amplada de l'objectiu o la distància virtual fins ell. A continuació, alguns
    consells basats en la llei de Fitts:

    \begin{itemize}
        \item Incrementar la mida de l'objectiu
        \item Mantenir els objectiu relacionats un a prop de l'altre i els elements oposats allunyats entre ells (sense oblidar els principis d'usabilitat)
        \item Filtres a prop del camp de cerca
        \item Als menus emergents, reduïr la distància a viatjar (opcions a prop de l'ubicació de l'opció del menú).
        \item Als menus pastís, no haurien d'existir oclusions (és difícil de dissenyar)
        \item Agrupar objectius sempre que es pugui
    \end{itemize}

    \hfill \break
    Per altra banda, per accelerar l'adquisició d'objectius:

    \begin{itemize}
        \item Objectius d'expansió dinàmica: Incrementar la mida dels objectius propers a l'apuntador, o incrementar la mida del cursor
        quan es troba a prop dels objectius (bubble cursor)
        \item Moviments de l'objectiu: Moure els objectius cap a l'usuari.
        \item Relació entre l'amplitud dels moviments de la mà real de l'usuari i la dels moviments del cursor virtual, fer el mapeig constant, 
        que depengui de la velocitat del mouse o que depengui de la posició del cursor. (Mac OSX i Windows utilitzen acceleració del ratolí). Un punt 
        negatiu és que no queda clar com afecta aquest mapeig a la percepció i la productivitat.
    \end{itemize}

    \subsection{Dispositius apuntadors}
        \subsubsection{Dispositius de control directe}
        Són tots aquells en els que es treballa directament amb la superfície de la pantalla.
        \begin{itemize}
            \item \textbf{Principals inconvenients}: Imprecisió de l'apuntat (per la qualitat de la pantalla i la mida de l'apuntador). Possibles solucions són 
            \textit{Land-on strategy} i \textit{Lift-off strategy}, on el click inicial crea un cursor.
            \item \textbf{Principals avantatges}: Les pantalles tàctils poden ser dissenyades sense parts mòvils i el poder tocar la pantalla amb múltiples dits permet
            l'entrada de dades més complexes i en conseqüència més opcions de manipulació de l'interfície (per exemple, zoom amb els dits)

        \end{itemize}
    
        \subsubsection{Dispositius de control indirecte}
        Són tots aquells on es treballa fora de la superfície de la interfície (es mapeja el moviment de l'usuari a un element d'apuntat virtual, per exemple el cursor).
        Les principals característiques són: alleuja la fatiga de les mans, s'elimina l'oclusió de la pantalla, el ratolí és el que mana a l'interacció.

    \subsection{Mecanografia i teclats}
    Existeixen múltiples disposicions de teclat a l'actualitat, però la predominant és sens dubte \textbf{QWERTY}. Es col·loquen les tecles que acostumen a emprar-se juntes
    amb bastanta distància física, per distribuïr la feina entre tots els dits i millorar així la nostra eficiencia. Però el principal problema és que no tothom escriu en
    anglès, que és el llenguatge pel qual es va fer aquesta distribució.

    \hfill \break
    Una altra distribució ergonòmica és \textbf{AZERTY}, optimitzada pel francès. Però, la principal alternativa a QWERTY és \textbf{DVORAK}, on les vocals es troben una 
    al costat de l'altra, de forma que les lletres més comunes es troben als llocs on descansen els dits. Aquesta distribució es va inventar amb l'objectiu de reduïr les
    distàncies de viatge entre tecles. De fet, ha permès millores de fins al 30\% i menys errors, però és molt difícil reemplaçar a QWERTY perquè té un nivell d'aceptació molt elevat,
    tothom hi està acostumbrat, a diferència de DVORAK.


    \hfill \break
    Les disposicions del teclat s'han de dissenyar de manera que:
    \begin{itemize}
        \item La càrrega sigui equilibrada per cada mà
        \item Es maximitzi la càrrega a la fila central
        \item Es maximitzi la freqüència d'alternar mà
        \item Es minimitzi la freqüència de usar el mateix dit de forma seguida
    \end{itemize}

    Pel que fa a les distribucions tàctils, els principals inconvenients són: la mida depèn de la pantalla, es requerex més atenció, i hi ha molts més errors. De fet, 
    aquest és el cas dels dispositius mòvils, on les principals distribucions són:
    \begin{itemize}
        \item \textbf{Minuum}: escriuen un o dos dits. Es comprimeixen les tres files d'un teclat en una sola.
        \item \textbf{Single finger gesture typing}: El dit travessa les lletres sense abandonar la superfície de la pantalla. Més cómode però no més ràpid que l'escriptura regular.
        \item \textbf{KALQ}: Optimitzat per escriure amb els dos polzes.
        \item \textbf{Two finger gesture typing}: escriptura amb dos dits (polzes), però el que coneixem de sempre, KALQ separa el teclat en dos.
    \end{itemize}

    Llavors, en un teclat virtual cal considerar per l'usabilitat els següents elements:
    \begin{itemize}
            \item Auto-correcció (model predictiu, i que l'usuari decideixi)
            \item Majúscules automàtiques (en les direccions mail deshabilitar-ho)
            \item Dissenys adequats pel tipus d'entrada (es poden fer dissenys dedicats)
            \item Suport per múltiples llenguatges (possibilitat de cambiar de llenguatge fàcilment)
    \end{itemize}

    \section{Tests d'Usabilitat}
    \subsection{Conceptes}
    L'usabilitat es tracta de la facilitat d'ús i acceptabilitat d'un sistema o producte per a una classe particular d'usuaris
    que realitzen activitats específiques en un entorn concret. Per ser útil, la usabilitat ha de ser específica.

    \hfill \break
    Però com fem els tests d'usabilitat? La facilitat d'ús és inversament proporcional al pes de la dificultat que tenen les persones 
    a l'utilitzar el programari.

    \subsection{Tests d'usabilitat}
    Hi ha dos grups principals de tests, separats pels seus objectius:
    \begin{itemize} 
        \item Per determinar problemes d'usabilitat, s'utilitza \textit{Iterative Testing}.
        \item Per mesurar el rendiment, on s'inclouen dues tasques fonamentals: el desenvolupament dels objectius 
        d'usabilitat i \textit{Iterative Testing} per determinar si el producte a prova compleix els objectius.
    \end{itemize}
    
    Hi ha molts tipus de tests d'usabilitat: informals o formals, remots o locals, el software evaluat pot variar, etc.

    \subsection{Tests formals d'usabilitat}
    \subsubsection {Ambient}
    Les proves formals d'usabilitat requereixen que es realitzin en un ambient controlat. Per exemple, per segons quin test es podria
    necessitar un conjunt d'aules insonoritzades o diferent tipus d'equipament i àrees (de participant, d'observador, de càmeres, etc)

    \subsubsection{Tasques i rols}
    El flux de treball per aquests tests es basa en tres fases: preparació, implementació i realització dels informes.

    Existeixen diferents rols a l'hora de fer les diferens tasques:
    \begin{itemize} 
        \item Administrador dels tests: És el que s'encarrega principalment de la fase de preparació.
        \item \textit{Briefer}
        \item Operador de càmera    
        \item Enregistrador de dades
        \item Operador d'ajuda
        \item Expert en el producte
        \item Persona experta en estadística
    \end{itemize}

    A la pipeline d'execució del test, desde la preparació fins la realització dels informes, tots aquests rols intervenen en determinades activitats. A les diapos apareix
    tot de forma molt detallada, però considero que no té importància.

    \subsection{Tests simplificats d'usabilitat}
    Són tests més senzills, en el sentit que els pot fer una persona sola sense la necessitat d'un gran grup.
    \subsubsection{Test d'usabilitat Guerrilla}
    Demanar a algú conegut d'utilitzar una interfície determinada i preguntar-li que opina, i fer preguntes de l'estil: Tú què faries?, per posteriorment analitzar les
    seves sensacions i propostes.
    \subsubsection{Usability testing on 10 cents a day}
    Similar a Guerrilla, es preparen algunes tasques a evaluar, i s'ofereix a algun company de treball la realització d'unes tasques determinades, capturant i gravant la
    seva interacció. Posteriorment fer un informe.
    \subsubsection{Testing remot}
    Igual que els tests tradicionals, però els participants dels tests es troben en ubicacions físiques diferents, poden fer els tests a casa. Els principals avantatges són: 
    més econòmic i fàcil, habitualment més ràpid. Els principals inconvenients són: no podem llegir el llenguatge corporal, és difícil decidir quan parlar/nteractuar.
    Hi ha dos tipus principals de testing remot: el no moderat, on els usuaris realitzen la tasca sols, i el moderat, on els usuaris tenen accés a un facilitador.
    \subsubsection{Avaluació heurística}
    Entre 3 i 5 experts d'usabilitat avaluen una aplicació o interfície. Pot ser ràpida i efectiva i és compatible amb els demés mètodes de testing.

    \subsection{Casos d'ús}
    Pel que fa a aquest apartat, a les diapositives es mostren exemples d'aplicacions/interfícies reals, a les quals se'ls hi aplica un test d'usabilitat determinat. És convenient simplement fer-hi
    una ullada.

    \section{Gràfiques per a la representació d'informació (Tema extra)}
    És important que les gràfiques que usem per representar informació siguin clares, concises i correctes.\\
    Alguns principis bàsics a seguir quan es construeixen gràfiques són:
    \begin{itemize}
        \item Hi ha d'haver una proporció alta de dades als gràfics $\rightarrow$ els punts de dades han de ser clarament visibles (que no els tapi títols, etc).
        \item S'ha d'utilitzar la gràfica correcte pel propòsit que tenim. No hem d'utilitzar gràfics circulars (\textit{Pie Chart}) per tot!! Hi ha tres tipus principals de gràfiques:
        \begin{itemize}
            \item Gràfiques de tendència (\textit{Trend Graphs}): útil per emfatitzar una evolució de cert paràmetre a través del temps
            \item Gràfiques de mida relativa (\textit{Relative Size Graphs}): gràfiques de barres, on totes les barres tenen la mateixa amplada, per notar diferència únicament en la longitud.
            \item Gràfiques de composició (\textit{Composition Graphs}): Aquí és on habitualment la gent empra de forma incorrecta els \textit{pie-charts}, ja que la gent no està acostumada a
            interpretar i comparar angles $\rightarrow$ una millor opció és utilitzar gràfics de barres.

        \end{itemize}
        \item El gràfic ha d'estar ben etiquetat i ha de tenir títol.
        \item És important pensar en la presentació general del gràfic. (escollir bones escales pels eixos, els gràfics de barres s'han d'anclar al valor 0, traçar línies per diferenciar grups, etc)
    \end{itemize}

    \hfill \break
    Pel que fa a alguns errors comuns:
    \begin{itemize}
        \item Tipus de gràfic incorrecte (tendències millor amb línies i composicions millor amb barres segmentades).
        \item Falta de text (falta alguna etiqueta o no hi ha títol).
        \item Escala inconsistent: l'escala no es constant a tot el gràfic
        \item Punt zero fora de lloc (no es troba a la part inferior)
        \item Efectes gràfics deficients: afegir ombres, efectes tridimensionals, etc, pot distorsionar el gràfic i afegir-hi poca informació o cap. En particular,
        else efectes 3D són molt deficients, ja que sovint dificulten més l'interpretació del gràfic, i pitjor si es troba inclinat.
        \item Confusió d'àrea i longitud
        \item Sense ajust per inflació. (Les quantitats en dòlars s'han d'ajustar per l'inflació, evitant comparacions errònees).
        \item Massa precisió: Donar molts dígits significatius és una tonteria, cal utilitzar una escala adequada.
    \end{itemize}

    En el document del tema extra, hi ha una sèrie d'exemples de gràfiques amb errors. Convé fer-hi una ullada. (Cal recordar que el terme ducks que es menciona sovint es correspon a elememts innecessaris
    que no aporten informació).

\end{document}









